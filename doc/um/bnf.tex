\chapter*{Appendix A - Formal syntax of RFSM programs}
\label{cha:bnf}

This appendix gives a BNF definition of the concrete syntax RFSM programs.

\medskip
The meta-syntax is conventional. Keywords are written in \textbf{boldface}.  Non-terminals are
enclosed in angle brackets ({\tt <} \ldots {\tt >}).  Vertical bars ({\tt |}) indicate
alternatives.  Constructs enclosed in brackets ({\tt [} \ldots {\tt ]}) are optional.
The notation $E^*$ (resp $E^+$) means zero (resp one) or more repetitions of $E$, separated by spaces.
The notation $E^*_x$ (resp $E^+_x$) means zero (resp one) or more repetitions of $E$, separated by
symbol $x$. Terminals \verb|lid| and \verb|uid| respectively designate identifiers
starting with a lowercase and uppercase letter. Terminal \verb|int| designates a positive or nul
integer.

\begin{rfsmgrammar}
\rfsmgramfunc{\rfsmprogram{}}& \rfsmgramdef & \rfsmgramstar{\rfsmtypedecl*{}} \\ & &
                                              \rfsmgramstar{\rfsmcstdecl*{}} \\ & &
                                              \rfsmgramstar{\rfsmfndecl*{}} \\ & &
                                              \rfsmgramstar{\rfsmfsmmodel*{}} \\ & &
                                              \rfsmgramstar{\rfsmfsmmodel*{}} \\ & &
                                              \rfsmgramstar{\rfsmglobal*{}} \\ & &
                                              \rfsmgramstar{\rfsmfsminst*{}}
                                              \rfsmEOF*{}
  \\& & \\

\rfsmgramfunc{\rfsmtypedecl{}}& \rfsmgramdef & \rfsmTYPE*{} \rfsmLID*{}
                                               \rfsmEQUAL*{} \rfsmtypeexpr*{}\\
  & \rfsmgrambar &\rfsmTYPE*{} \rfsmLID*{} \rfsmEQUAL*{} \rfsmENUM*{}
                  \rfsmbraced{\rfsmgramseplist{\rfsmCOMMA*{}}{\rfsmUID*{}}}\\
  & \rfsmgrambar &\rfsmTYPE*{} \rfsmLID*{} \rfsmEQUAL*{} \rfsmRECORD*{}
                  \rfsmLBRACE*{}
                  \rfsmgramsepnelist{\rfsmCOMMA*{}}{\rfsmrecordfield*{}}
                  \rfsmRBRACE*{}
  \\& & \\

\rfsmgramfunc{\rfsmrecordfield{}}& \rfsmgramdef & \rfsmLID*{} \rfsmCOLON*{}
                                                  \rfsmtypeexpr*{}
  \\& & \\

\rfsmgramfunc{\rfsmcstdecl{}}& \rfsmgramdef & \rfsmCONSTANT*{} \rfsmLID*{}
                                              \rfsmCOLON*{} \rfsmtypeexpr*{}
                                              \rfsmEQUAL*{} \rfsmconst*{}
  \\& & \\

\rfsmgramfunc{\rfsmfndecl{}}& \rfsmgramdef & \rfsmFUNCTION*{} \rfsmLID*{}
                                             \rfsmLPAREN*{}
                                             \rfsmgramseplist{\rfsmCOMMA*{}}{\rfsmfarg*{}}
                                             \rfsmRPAREN*{} \rfsmCOLON*{}
                                             \rfsmtypeexpr*{} \rfsmLBRACE*{}
                                             \rfsmRETURN*{} \rfsmexpr*{}
                                             \rfsmRBRACE*{}
  \\& & \\

\rfsmgramfunc{\rfsmfarg{}}& \rfsmgramdef & \rfsmLID*{} \rfsmCOLON*{}
                                           \rfsmtypeexpr*{}
  \\& & \\

\rfsmgramfunc{\rfsmfsmmodel{}}& \rfsmgramdef & \rfsmFSM*{} \rfsmMODEL*{}
                                               \rfsmid*{}
                                               \rfsmgramopt{\rfsmparams*{}}
                                               \rfsmLPAREN*{}
                                               \rfsmgramseplist{\rfsmCOMMA*{}}{\rfsmio*{}}
                                               \rfsmRPAREN*{} \rfsmLBRACE*{} \\ & &
                                               \rfsmSTATES*{} \rfsmCOLON*{}
                                               \rfsmgramseplist{\rfsmCOMMA*{}}{\rfsmstate*{}}
                                               \rfsmSEMICOLON*{} \\ & &
                                               \rfsmgramopt{\rfsmvars*{}} \\ & &
                                               \rfsmTRANS*{} \rfsmCOLON*{}
                                               \rfsmgramseplist{\rfsmCOMMA*{}}{\rfsmtransition*{}}
                                               \rfsmSEMICOLON*{} \\ & &
                                               \rfsmITRANS*{} \rfsmCOLON*{}
                                               \rfsmitransition*{}
                                               \rfsmSEMICOLON*{} \\ & &
                                               \rfsmRBRACE*{}
  \\& & \\

  \rfsmgramfunc{\rfsmstate{}}& \rfsmgramdef & \rfsmUID*{}
                                              \rfsmgramopt{\rfsmWHERE \rfsmgramsepnelist{\rfsmAND{}}{\rfsmoutpval*{}}}

  \\& & \\
\rfsmgramfunc{\rfsmoutpval{}}& \rfsmgramdef & \rfsmLID*{} \rfsmEQUAL*{} \rfsmscalarconst*{}


  \\& & \\

\rfsmgramfunc{\rfsmparams{}}& \rfsmgramdef & \rfsmLT*{}
                                             \rfsmgramseplist{\rfsmCOMMA*{}}{\rfsmparam*{}}
                                             \rfsmGT*{}
  \\& & \\

\rfsmgramfunc{\rfsmparam{}}& \rfsmgramdef & \rfsmLID*{} \rfsmCOLON*{}
                                            \rfsmtypeexpr*{}
  \\& & \\

\rfsmgramfunc{\rfsmio{}}& \rfsmgramdef & \rfsmIN*{} \rfsmiodesc*{}\\
  & \rfsmgrambar &\rfsmOUT*{} \rfsmiodesc*{}\\
  & \rfsmgrambar &\rfsmINOUT*{} \rfsmiodesc*{}
  \\& & \\

\rfsmgramfunc{\rfsmiodesc{}}& \rfsmgramdef & \rfsmLID*{} \rfsmCOLON*{}
                                             \rfsmtypeexpr*{}
  \\& & \\

\rfsmgramfunc{\rfsmvars{}}& \rfsmgramdef & \rfsmVARS*{} \rfsmCOLON*{}
                                           \rfsmgramseplist{\rfsmCOMMA*{}}{\rfsmvar*{}}
                                           \rfsmSEMICOLON*{}
  \\& & \\

\rfsmgramfunc{\rfsmvar{}}& \rfsmgramdef & \rfsmgramsepnelist{\rfsmCOMMA*{}}{\rfsmLID*{}}
                                          \rfsmCOLON*{} \rfsmtypeexpr*{}
  \\& & \\

\rfsmgramfunc{\rfsmtransition{}}& \rfsmgramdef & \rfsmrulepfx*{}
                                                 \rfsmUID*{}
                                                 \rfsmARROW*{}
                                                 \rfsmUID*{}
                                                 \rfsmcondition*{}
                                                 \rfsmgramopt{\rfsmactions*{}}
  \\& & \\

  \rfsmgramfunc{\rfsmrulepfx{}}& \rfsmgramdef & \rfsmBAR*{} \rfsmgrambar \rfsmEMARK*{}

  \\& & \\

\rfsmgramfunc{\rfsmcondition{}}& \rfsmgramdef & \rfsmON*{} \rfsmLID*{} \rfsmgramopt{\rfsmguards*{}} \\

  \\& & \\

\rfsmgramfunc{\rfsmguards{}}& \rfsmgramdef & \rfsmWHEN*{} \rfsmgramsepnelist{\rfsmDOT*{}}{\rfsmexpr*{}}

  \\& & \\

\rfsmgramfunc{\rfsmactions{}}& \rfsmgramdef & \rfsmWITH*{} \rfsmgramsepnelist{\rfsmSEMICOLON*{}}{\rfsmaction*{}}
  \\& & \\

\rfsmgramfunc{\rfsmaction{}}& \rfsmgramdef & \rfsmLID*{}\\
  & \rfsmgrambar &\rfsmlhs*{} \rfsmCOLEQ*{} \rfsmexpr*{}
  \\& & \\

\rfsmgramfunc{\rfsmlhs{}}& \rfsmgramdef & \rfsmLID*{}\\
  & \rfsmgrambar &\rfsmLID*{} \rfsmLBRACKET*{} \rfsmexpr*{} \rfsmRBRACKET*{}\\
  & \rfsmgrambar &\rfsmLID*{} \rfsmLBRACKET*{} \rfsmexpr*{} \rfsmCOLON*{}
                  \rfsmexpr*{} \rfsmRBRACKET*{}\\
  & \rfsmgrambar &\rfsmLID*{} \rfsmDOT*{} \rfsmLID*{}
  \\& & \\

  \rfsmgramfunc{\rfsmitransition{}}& \rfsmgramdef & \rfsmBAR*{} \rfsmARROW*{} \rfsmUID*{} \rfsmgramopt{\rfsmactions*{}}
                                                  
  \\& & \\

\rfsmgramfunc{\rfsmglobal{}}& \rfsmgramdef & \rfsmINPUT*{} \rfsmid*{}
                                             \rfsmCOLON*{} \rfsmtypeexpr*{}
                                             \rfsmEQUAL*{} \rfsmstimuli*{}\\
  & \rfsmgrambar &\rfsmOUTPUT*{}
                  \rfsmgramsepnelist{\rfsmCOMMA*{}}{\rfsmid*{}} \rfsmCOLON*{}
                  \rfsmtypeexpr*{}\\
  & \rfsmgrambar &\rfsmSHARED*{}
                  \rfsmgramsepnelist{\rfsmCOMMA*{}}{\rfsmid*{}} \rfsmCOLON*{}
                  \rfsmtypeexpr*{}
  \\& & \\

\rfsmgramfunc{\rfsmstimuli{}}& \rfsmgramdef & \rfsmPERIODIC*{} \rfsmLPAREN*{}
                                              \rfsmINT*{} \rfsmCOMMA*{}
                                              \rfsmINT*{} \rfsmCOMMA*{}
                                              \rfsmINT*{} \rfsmRPAREN*{}\\
  & \rfsmgrambar &\rfsmSPORADIC*{}
                  \rfsmparen{\rfsmgramseplist{\rfsmCOMMA*{}}{\rfsmINT*{}}}\\
  & \rfsmgrambar &\rfsmVALUECHANGES*{}
                  \rfsmparen{\rfsmgramseplist{\rfsmCOMMA*{}}{\rfsmvaluechange*{}}}
  \\& & \\

\rfsmgramfunc{\rfsmvaluechange{}}& \rfsmgramdef & \rfsmINT*{} \rfsmCOLON*{}
                                                  \rfsmstimconst*{}
  \\& & \\

\rfsmgramfunc{\rfsmfsminst{}}& \rfsmgramdef & \rfsmFSM*{} \rfsmid*{}
                                              \rfsmEQUAL*{} \rfsmid*{}
                                              \rfsmgramopt{\rfsmLT*{}
                                              \rfsmgramsepnelist{\rfsmCOMMA*{}}{\rfsmparamvalue*{}}
                                              \rfsmGT*{}}
                                              \rfsmparen{\rfsmgramseplist{\rfsmCOMMA*{}}{\rfsmid*{}}}
  \\& & \\

\rfsmgramfunc{\rfsmparamvalue{}}& \rfsmgramdef & \rfsmscalarconst*{}\\
  & \rfsmgrambar &\rfsmLID*{}

  \\& & \\

\rfsmgramfunc{\rfsmtypeexpr{}}& \rfsmgramdef & \rfsmTYEVENT*{}\\
  & \rfsmgrambar &\rfsmTYINT*{} \rfsmintannot*{}\\
  & \rfsmgrambar &\rfsmTYFLOAT*{}\\
  & \rfsmgrambar &\rfsmTYCHAR*{}\\
  & \rfsmgrambar &\rfsmTYBOOL*{}\\
  & \rfsmgrambar &\rfsmLID*{}\\
  & \rfsmgrambar &\rfsmtypeexpr*{} \rfsmTYARRAY*{} \rfsmLBRACKET*{}
                  \rfsmarraysize*{} \rfsmRBRACKET*{}
  \\& & \\

\rfsmgramfunc{\rfsmintannot{}}& \rfsmgramdef & \rfsmgrameps \\
  & \rfsmgrambar &\rfsmLT*{} \rfsmtypesize*{} \rfsmGT*{}\\
  & \rfsmgrambar &\rfsmLT*{} \rfsmtypesize*{} \rfsmCOLON*{}
                  \rfsmtypesize*{} \rfsmGT*{}
  \\& & \\

\rfsmgramfunc{\rfsmarraysize{}}& \rfsmgramdef & \rfsmtypesize*{}
  \\& & \\

\rfsmgramfunc{\rfsmtypesize{}}& \rfsmgramdef & \rfsmINT*{}\\
  & \rfsmgrambar &\rfsmLID*{}\\

  \\& & \\

\rfsmgramfunc{\rfsmexpr{}}& \rfsmgramdef & \rfsmsimpleexpr*{}\\
  & \rfsmgrambar &\rfsmexpr*{} \rfsmPLUS*{} \rfsmexpr*{}\\
  & \rfsmgrambar &\rfsmexpr*{} \rfsmMINUS*{} \rfsmexpr*{}\\
  & \rfsmgrambar &\rfsmexpr*{} \rfsmTIMES*{} \rfsmexpr*{}\\
  & \rfsmgrambar &\rfsmexpr*{} \rfsmDIV*{} \rfsmexpr*{}\\
  & \rfsmgrambar &\rfsmexpr*{} \rfsmMOD*{} \rfsmexpr*{}\\
  & \rfsmgrambar &\rfsmexpr*{} \rfsmEQUAL*{} \rfsmexpr*{}\\
  & \rfsmgrambar &\rfsmexpr*{} \rfsmNOTEQUAL*{} \rfsmexpr*{}\\
  & \rfsmgrambar &\rfsmexpr*{} \rfsmGT*{} \rfsmexpr*{}\\
  & \rfsmgrambar &\rfsmexpr*{} \rfsmLT*{} \rfsmexpr*{}\\
  & \rfsmgrambar &\rfsmexpr*{} \rfsmGTE*{} \rfsmexpr*{}\\
  & \rfsmgrambar &\rfsmexpr*{} \rfsmLTE*{} \rfsmexpr*{}\\
  & \rfsmgrambar &\rfsmexpr*{} \rfsmLAND*{} \rfsmexpr*{}\\
  & \rfsmgrambar &\rfsmexpr*{} \rfsmLOR*{} \rfsmexpr*{}\\
  & \rfsmgrambar &\rfsmexpr*{} \rfsmLXOR*{} \rfsmexpr*{}\\
  & \rfsmgrambar &\rfsmexpr*{} \rfsmSHR*{} \rfsmexpr*{}\\
  & \rfsmgrambar &\rfsmexpr*{} \rfsmSHL*{} \rfsmexpr*{}\\
  & \rfsmgrambar &\rfsmexpr*{} \rfsmFPLUS*{} \rfsmexpr*{}\\
  & \rfsmgrambar &\rfsmexpr*{} \rfsmFMINUS*{} \rfsmexpr*{}\\
  & \rfsmgrambar &\rfsmexpr*{} \rfsmFTIMES*{} \rfsmexpr*{}\\
  & \rfsmgrambar &\rfsmexpr*{} \rfsmFDIV*{} \rfsmexpr*{}\\
  & \rfsmgrambar &\rfsmsubtractive*{} \rfsmexpr*{}\\
  & \rfsmgrambar &\rfsmLID*{} \rfsmLBRACKET*{} \rfsmexpr*{} \rfsmRBRACKET*{}\\
  & \rfsmgrambar &\rfsmLID*{} \rfsmLBRACKET*{} \rfsmexpr*{} \rfsmCOLON*{}
                  \rfsmexpr*{} \rfsmRBRACKET*{}\\
  & \rfsmgrambar &\rfsmLID*{} \rfsmLPAREN*{}
                  \rfsmgramseplist{\rfsmCOMMA*{}}{\rfsmexpr*{}}
                  \rfsmRPAREN*{}\\
  & \rfsmgrambar &\rfsmLID*{} \rfsmDOT*{} \rfsmLID*{}\\
  & \rfsmgrambar &\rfsmexpr*{} \rfsmQMARK*{} \rfsmexpr*{} \rfsmCOLON*{}
                  \rfsmexpr*{}\\
  & \rfsmgrambar &\rfsmexpr*{} \rfsmCOLONCOLON*{} \rfsmtypeexpr*{}
  \\& & \\

\rfsmgramfunc{\rfsmsimpleexpr{}}& \rfsmgramdef & \rfsmLID*{}\\
  & \rfsmgrambar &\rfsmUID*{}\\
  & \rfsmgrambar &\rfsmscalarconst*{}\\
  & \rfsmgrambar &\rfsmLPAREN*{} \rfsmexpr*{} \rfsmRPAREN*{}
  \\& & \\

\rfsmgramfunc{\rfsmsubtractive{}}& \rfsmgramdef & \rfsmMINUS*{}\\
  & \rfsmgrambar &\rfsmFMINUS*{}
  \\& & \\

\rfsmgramfunc{\rfsmscalarconst{}}& \rfsmgramdef & \rfsmINT*{}\\
  & \rfsmgrambar &\rfsmBOOL*{}\\
  & \rfsmgrambar &\rfsmFLOAT*{}\\
  & \rfsmgrambar &\rfsmCHAR*{}
  \\& & \\

\rfsmgramfunc{\rfsmconst{}}& \rfsmgramdef & \rfsmscalarconst*{}\\
  & \rfsmgrambar &\rfsmarrayconst*{}\\
  \\& & \\

\rfsmgramfunc{\rfsmarrayconst{}}& \rfsmgramdef & \rfsmLBRACKET*{}
                                                 \rfsmgramsepnelist{\rfsmCOMMA*{}}{\rfsmconst*{}}
                                                 \rfsmRBRACKET*{}
  \\& & \\

\rfsmgramfunc{\rfsmstimconst{}}& \rfsmgramdef & \rfsmscalarconst*{}\\
  & \rfsmgrambar &\rfsmscalarconst*{} \rfsmCOLONCOLON*{} \rfsmtypeexpr*{}\\
  & \rfsmgrambar & \rfsmUID*{}\\
  & \rfsmgrambar & \rfsmrecordconst*{}\
  \\& & \\

\rfsmgramfunc{\rfsmrecordconst{}}& \rfsmgramdef & \rfsmLBRACE*{}
                                                  \rfsmgramsepnelist{\rfsmCOMMA*{}}{\rfsmrecordfieldconst*{}}
                                                  \rfsmRBRACE*{}
  \\& & \\

\rfsmgramfunc{\rfsmrecordfieldconst{}}& \rfsmgramdef & \rfsmLID*{}
                                                       \rfsmEQUAL*{}
                                                       \rfsmstimconst*{}
  \\& & \\

\rfsmgramfunc{\rfsmid{}}& \rfsmgramdef & \rfsmLID*{}\\
  & \rfsmgrambar &\rfsmUID*{}
  \\& & \\

\end{rfsmgrammar}

%%% Local Variables: 
%%% mode: latex
%%% TeX-master: "rfsm_rm"
%%% End: 


%%% Local Variables:
%%% mode: latex
%%% TeX-master: "rfsm"
%%% End:
