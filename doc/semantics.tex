\chapter{Formal semantics}
\label{cha:semantics}

\newcommand{\truev}{\mathsf{T}}
\newcommand{\tuple}[1]{\langle#1\rangle}
\newcommand{\ttuple}[2]{\langle#1, #2\rangle}
\newcommand{\tttuple}[3]{\langle#1, #2, #3\rangle}
\newcommand{\ttttuple}[4]{\langle#1, #2, #3, #4\rangle}
\newcommand{\tttttuple}[5]{\langle#1, #2, #3, #4, #5\rangle}
\newcommand{\ttttttuple}[6]{\langle#1, #2, #3, #4, #5, #6\rangle}
\newcommand{\tttttttuple}[7]{\langle#1, #2, #3, #4, #5, #6\rangle}
\newcommand{\ttttttttuple}[8]{\langle#1, #2, #3, #4, #5, #6\rangle}
\newcommand{\tuplen}[1]{\langle#1_1,\ldots,#1_n\rangle}
\newcommand{\tuplez}{\langle\rangle}
\newcommand{\cupp}[3]{\displaystyle{\bigcup_{#1}^{#2}}~#3}
\newcommand{\capp}[3]{\displaystyle{\bigcap_{#1}^{#2}}~#3}
\newcommand{\oplusn}[3]{\displaystyle{\bigoplus_{#1}^{#2}}~#3}
\newcommand{\valred}[2]{\rho_{#2}(#1)}
\newcommand{\emptyseq}{\langle\rangle}
\newcommand{\sequ}[1]{\langle#1\rangle}
\newcommand{\ssequ}[2]{\langle#1; #2\rangle}
\newcommand{\sssequ}[3]{\langle#1; #2; #3\rangle}
\newcommand{\sequn}[1]{\langle#1_1;\ldots;#1_n\rangle}
\newcommand{\sequm}[2]{\langle#1_1;\ldots;#1_#2\rangle}
\newcommand{\squn}[1]{#1_1,\ldots,#1_n}

\newcommand{\delt}[4]{\ttttuple{#1}{#2}{#3}{#4}}
%\newcommand{\delt}[4]{(#1,#2)=(#4,#3)}
\newcommand{\trans}[3]{#1 \xrightarrow{#2} #3}
\newcommand{\transs}[4]{#1 \xrightarrow[#3]{#2} #4}
\newcommand\doubleplus{+\kern-1.3ex+\kern0.8ex}

\newcommand{\larrow}{\xrightarrow}
\newcommand{\seqn}[3]{#3_#1,\ldots,#3_#2}
\newcommand{\cuppn}[1]{\cupp{i=1}{n}{#1}}
\newcommand{\setn}[1]{\{#1_1,\ldots,#1_n\}}
\newcommand{\mm}{\mathcal{M}}
\newcommand{\ssigma}{\overline{\sigma}}
\newcommand{\ssigm}{\overline{s}}
\newcommand{\eval}[2]{\mathcal{E}_{#1}\llbracket #2 \rrbracket}
\newcommand{\falln}[3]{\forall #1\in\{#2,\ldots,#3\}\quad}
% \newcommand{\falln}[3]{\forall #1=#2,\ldots,#3\quad}
\newcommand{\semfn}[3]{\mathcal{#1}_{#2}\llbracket #3 \rrbracket}
\newcommand{\vars}{\mathcal{V}}
\newcommand{\env}{\Gamma}
\newcommand{\expr}{\mathsf{e}}
\newcommand{\cupdot}{\mathbin{\mathaccent\cdot\cup}}

\vspace{-5mm}
We give formal \emph{static} and \emph{dynamic} semantics for a simplified version of the
\textsc{Rfsm} language, called \textsc{Core Rfsm}. Compared to the ``full'' \textsc{Rfsm} language,
it lacks type, constant and function declarations, state valuation and has only basic types. 
Its abstract syntax is described below. We note $X^*$ (resp. $X^+$) the repetition of 0
(resp. 1) or more $X$. The syntax of expressions is deliberately not explicited here. 

\newcommand{\tm}[1]{\mathtt{#1}}
\newcommand{\ut}[1]{\emph{#1}}
\newcommand{\cat}[1]{\text{#1}}

\bigskip
$  \begin{array}{lll}
    \ut{program} ::= & \tm{program}~\ut{fsm\_model}^+~\ut{io\_decl}^+~\ut{fsm\_inst}^+ & \\
    \\
    \ut{fsm\_model}  ::= & \tm{fsm\ model}~\cat{id}~\ut{inp}^*~\ut{outp}^*~\ut{state}^+~\ut{var}^*~\ut{trans}^+~\ut{itrans} &\\
    \\
    \ut{state}  ::= &  \cat{id} \\
    \ut{inp},\ut{outp},\ut{var}  ::= &  \cat{id}~\tm{:}~\ut{typ} & \\
    \ut{trans} ::= & \ttttuple{\cat{id}}{\ut{cond}}{\ut{action}^*}{\cat{id}} & \ttttuple{\emtxt{src
                                                                               state}}{\ut{cond}}{\ut{actions}}{\emtxt{dst
                                                                               state}} \\
    \ut{cond} ::= & \ttuple{\cat{id}}{\ut{guard}^*} & \ttuple{\emtxt{triggering even}}{\ut{guards}} \\
    \ut{guard} ::= & \ut{expr} & \emtxt{boolean expression} \\
    \ut{action} ::= & ~|\ \cat{id} & \emtxt{emit event} \\
                    & ~|\ \cat{id}~\tm{:=}~\ut{expr} & \emtxt{update local, shared or output
                                                       variable}\\
    \\
    \ut{io\_decl}  ::= & \ut{io\_cat}~\tm{id}~\tm{:}~\ut{typ} &\\
    \ut{io\_cat} ::= & \tm{input} ~|~ \tm{input} ~|~ \tm{shared} & \\
    \\
    \ut{fsm\_inst} ::= & \tm{fsm}~\cat{id}~\ut{i}^*~\ut{o}^* & \emtxt{model},\ \emtxt{IO bindings}\\
    \\
    \ut{typ} ::= & \tm{event}~|~\tm{int}~|~\tm{bool} \\
  \end{array} $

\section{Common definitions}
\label{sec:general-definitions}

Both the static and static semantics will use \emph{environments}. 
An \textbf{environment} is a (partial) map from \emph{names} to \emph{values}.
If $\env$ is an environment and $x$ a name, we will, classically, note
\begin{itemize}
\item $x \in \env$ if $x \in \text{dom}(\env)$,
\item $\env(x)$ the value mapped to $x$ in $\env$ ($\env(x)=\bot$ if $x \not\in \env$), 
% \item $[x \mapsto v]$ the singleton environment mapping $x$ to $v$,
\item $\env[x \mapsto v]$ the environment that maps $x$ to $v$ and behaves like $\env$
otherwise (possibly overriding an existing mapping of $x$),
\item $\emptyset$ the empty environment,
% \item $\env \oplus \env'$ the environment obtained by coherent merging of $\env$ and $\env'$,
%   \emph{i.e.}
%   $$
%   (\env \oplus \env')(x)=\begin{cases}
%                            v \qquad\text{if}\ \env(x)=v \wedge \env'(x)=\bot \vee \env(x)=\bot \wedge \env'(x)=v \\
%                            \bot \qquad\text{if}\ \env(x)=\bot \wedge \env'(x)=\bot \vee \env(x)=v
%                            \wedge \env'(x)=v' \wedge v \not= v'
%                          \end{cases}
%   $$
\end{itemize}

\section{Static semantics}
\label{sec:static-semantics}

The static interpretation of a \textsc{Core Rfsm} program is a pair

\begin{equation*}
  \mathcal{H} = \ttuple{M}{C}
\end{equation*}

where
\begin{itemize}
\item $M$ is a set of \textbf{automata},
\item $C$ is a \textbf{context}.
\end{itemize}

\medskip\step
A \textbf{context} is a 6-tuple $\ttttttuple{I_e}{I_v}{O_e}{O_v}{H_e}{H_v}$ where
\begin{itemize}
\item $I_e$ (resp. $O_e$, $H_e$) is the set of global inputs (resp. outputs, shared values) with an \emph{event} type,
\item $I_v$ (resp. $O_v$, $H_v$) is the set of global inputs (resp. outputs, shared values) with a
  \emph{non-event} type.
\end{itemize}


\medskip\step
An \textbf{automaton} $\mu \in M$ is a 3-tuple 

\begin{equation*}
  \mu = \tttuple{\mm}{q}{\vars}
\end{equation*}

where
\begin{itemize}
\item $\mm$ is the associated (static) model,
\item $q$ its current state,
\item $\vars$ an environment giving the current value of its local variables.
\end{itemize}

\medskip\step
A \textbf{model} $\mm$ is a 6-tuple

\begin{equation*}
  \mm = \ttttttuple{Q}{I}{O}{V}{T}{\tau_0}
\end{equation*}

where

\begin{itemize}
\item $Q$ is a (finite) set of \emph{states},
\item $I$ and $O$ are environments respectively mapping input and output names to types,
\item $V$ is an environment mapping local variable names to types,
\item $I_e = \{ x \in I ~|~ I(x)=\tm{event} \}$ and $I_v = \{ x \in I ~|~ I(x)\not=\tm{event} \}$,
\item $O_e = \{ x \in O ~|~ O(x)=\tm{event} \}$ and $O_v = \{ x \in O ~|~ O(x)\not=\tm{event} \}$,
%  each $v_i \in V$ taking its values in a finite domain $D_i$,
\item $T \subset Q \times C \times \mathcal{S}(A) \times Q$ is a set of \textbf{transitions}, where
  \begin{itemize}
  \item $C=I_e \times 2^{\mathcal{B}(I_v \cup V)}$,
  \item $\mathcal{B}(E)$ is the set of \emph{boolean expressions} built from a set of variables $E$ and the
  classical boolean operators\footnote{This set can be formally  derived from the abstract syntax.},
  \item $\mathcal{S}(A)$ is the set of \emph{sequences} built from elements of the set $A$, where a
    \emph{sequence} $\vec{a}$ is an ordered collection $\sequn{a}$\footnote{For example, if
      $A=\{1,2\}$, then
      $\mathcal{S}(A)=\{\emptyseq,\sequ{1},\sequ{2},\ssequ{1}{1},\ssequ{1}{2},\ssequ{2}{1},\ssequ{2}{2},\sssequ{1}{2}{1},\ldots\}$,
      where $\tuplez$ denotes the empty sequence.},
  \item $A=\mathcal{U}(O_v \cup V, I_v \cup V) \cup O_e$, 
\item $\mathcal{U}(E,E')$ is the set of
  \emph{assignations} of variables taken in a set $E$ by \emph{expressions} built from a set of
  variable $E'$ and the classical boolean and arithmetic operators and constants\footnote{Again,
this can be formally derived from the abstract syntax.}.
  \end{itemize}
\item $\tau_0 \in Q \times \mathcal{S}({\mathcal{U}(O_v \cup V,\emptyset)})$ is the \textbf{initial transition}.
\end{itemize}

\medskip
Having $\tau=(q,c,\vec{a},q')$ in $T$ means that there's a transition from state $q$ to state
$q'$ enabled by the condition $c$ and triggering a (possibly empty) sequence $\vec{a}$, where
\begin{itemize}
\item the condition $c \in C$ is made of
\begin{itemize}
\item a trigerring event $e \in I_e$,
\item a (possibly empty) set of boolean expressions (guards), involving inputs having a non-event
  type or local variables,
\end{itemize}
\item the actions in $\vec{a}$ consist either in the emission of an event or the modification of an output or
local variable.
\end{itemize}

\medskip The initial transition $\tau_0$ consists in a state (the initial state) and a (possibly
empty) sequence of initial actions. Contrary to actions associated to ``regular'' transitions,
initial actions cannot not emit events and the
assigned values cannot depend on inputs or local variables.
  
% \medskip Because we want the described systems to be \emph{reactive} and \emph{deterministic}, for
% each state $q \in Q$ and condition $c \in C$, there should be exactly one transition
% $t=(q,c,a,q') \in T$\footnote{This condition can be enforced, if not the case, by adding implicit
%   transitions looping on the current state and with an empty set of actions.}. In this case, the set
% $T$ can by replaced by a (total) \emph{function} $\delta$ defined as follows~:

% \begin{center}
%   $\delta: Q \times C \rightarrow Q \times 2^A$ \\
%   $\delta(q,c)=(q',a)\quad \text{iff}\quad (q,c,a,q') \in T$.
% \end{center}

% \medskip
% We will sometimes write
% \begin{itemize}
% \item $\delt{q}{c}{a}{q'}$ as $\trans{q}{c/a}{q'}$ and
% \item $\delt{q}{c}{\emptyset}{q'}$ as $\trans{q}{c}{q'}$.
% \end{itemize}

\medskip \textbf{Example}. The model of the automaton depicted below\footnote{This model, a
  calibrated pulse generator has been introduced in Fig.~\ref{fig:genimp-xxx} (Chap.~\ref{cha:}).}
can be formally described as $\mathcal{M}=\ttttttuple{Q}{I}{O}{V}{T}{\tau_0}$ where~:

\begin{tabular}[c]{cc}
  \begin{minipage}[b]{0.5\linewidth}
\begin{itemize}
\item $Q = \{ E0, E1 \}$
\item $I = \{ h \mapsto \mathsf{event}, e \mapsto \mathsf{bool} \}$
\item $O = \{ s \mapsto \mathsf{bool} \}$
\item $V = \{ k \mapsto \mathsf{bool} \}$
\item $T = \{\\
  \delt{E0}{\ttuple{H}{\{e=0\}}}{\emptyseq}{E0},\\
  \delt{E0}{\ttuple{H}{\{e=1\}}}{\ssequ{s \gets 1}{k \gets 1}}{E1},\\
  \delt{E1}{\ttuple{H}{\{k<3\}}}{\sequ{k \gets k+1}}{E1},\\
  \delt{E1}{\ttuple{H}{\{k=3\}}}{\sequ{s \gets 0}}{E0} \}$
\item $\tau_0 = \ttuple{E0}{\sequ{s \gets 0}}$
\end{itemize}
  \end{minipage} &
   \includegraphics[width=0.25\linewidth]{figs/gensig-model}
\end{tabular}

\newpage
\subsection*{Rules}
\label{sec:static-rules}

\newcommand{\senv}[1]{\Gamma_{\!\mathsf{#1}}}

\step Rule \textsc{Program} gives the static interpretation of a program. The static environment
$\senv{M}$ (resp. $\senv{I}$) records the (typed) declarations of models (resp. IOs). 

\infrule[Program]
{\ut{fsm\_model}^+ \xrightarrow{} \senv{M} \\
 \ut{io\_decl}^+ \xrightarrow{} \senv{I} \\
 \senv{M},\senv{I} \vdash\ \ut{fsm_inst}^+ \xrightarrow{} M \\
 C=\mathcal{L}(\senv{I})}
{\tm{program}~\ut{fsm\_model}^+~\ut{io\_decl}^+~\ut{fsm\_inst}^+ \xrightarrow{} M, C}

The $\mathcal{L}$ function builds a static context $C$ from the IO environment $\senv{I}$~:
\begin{eqnarray*}
\mathcal{L}(\senv{I}) = \ttttttuple{I_e}{I_v}{O_e}{O_v}{H_e}{H_v}
\end{eqnarray*}
\noindent
where
\begin{eqnarray*}
I_e=\{x \in \text{dom}(\senv{I}) ~|~ \senv{I}(x)=\ttuple{\cat{input}}{\tm{event}}\} &
I_v=\{x \in \text{dom}(\senv{I}) ~|~ \senv{I}(x)=\ttuple{\cat{input}}{\tau},\ \tau\not=\tm{event}\}\\
O_e=\{x \in \text{dom}(\senv{I}) ~|~ \senv{I}(x)=\ttuple{\cat{output}}{\tm{event}}\} &
O_v=\{x \in \text{dom}(\senv{I}) ~|~ \senv{I}(x)=\ttuple{\cat{output}}{\tau},\ \tau\not=\tm{event}\}\\
H_e=\{x \in \text{dom}(\senv{I}) ~|~ \senv{I}(x)=\ttuple{\cat{shared}}{\tm{event}}\} &
H_v=\{x \in \text{dom}(\senv{I}) ~|~ \senv{I}(x)=\ttuple{\cat{shared}}{\tau},\ \tau\not=\tm{event}\}
\end{eqnarray*}

\medskip\step
Rule \textsc{Models} gives the interpretation of model declarations, giving an environment
$\senv{M}$. 

\infrule[Models]
{\falln{i}{1}{n}\quad \senv{M}^{i-1},\ \ut{fsm\_model}_i \xrightarrow{} \senv{M}^i \\
  \senv{M}^0 = \emptyset\quad
  \senv{M} = \senv{M}^n}
{\squn{\ut{fsm\_model}} \xrightarrow{} \senv{M}}

\medskip\step
Rule \textsc{Model} gives the interpretation of a single model declaration. It just records the
corresponding description in the environment $\senv{M}$, after performing some sanity checks, using
the $\mathsf{valid\_model}$ function, not detailed here. This function checks that~:
\begin{itemize}
\item all variable names occuring in guards are listed as input or local variable,
\item all expressions occuring in the guards of a transition have type $\mathsf{bool}$,
\item \ldots \todo{TBC}
\end{itemize}

\infrule[Model]
{\mm=\ttttttuple{Q}{I}{O}{V}{T}{\tau_0}\quad
  \mathsf{valid\_model}(\mm)}
{\tm{fsm\ model}~\cat{id}~I~O~Q~V~T~\tau_0 \xrightarrow{} \senv{M}[\cat{id}\mapsto\mm]}

\medskip\step
Rules \textsc{IOs} and \textsc{IO} give the interpretation of IO declarations, producing an environment
$\senv{I}$ binding names to a pair $\ttuple{\ut{io\_cat}}{\ut{typ}}$.

\infrule[IOs]
{\falln{i}{1}{n}\quad \senv{I}^{i-1},\ \ut{io_decl}_i \xrightarrow{} \senv{I}^i \\
 \senv{I}^0 = \emptyset\quad
  \senv{I} = \senv{I}^n}
{\squn{\ut{io\_decl}} \xrightarrow{} \senv{I}}

\infrule[IO]
{}
{\senv{I},\ \ut{cat}~\tm{id}~\tm{:}~\ut{typ} \xrightarrow{} \senv{I}[\tm{id}\mapsto
  \ttuple{\ut{cat}}{\ut{typ}}]}

\medskip\step
Rules \textsc{Insts} gives the interpretation of FSM instance declarations.

\infrule[Insts]
{\falln{i}{1}{n}\quad \senv{M},\senv{I} \vdash\ \ut{fsm\_inst}_i \xrightarrow{} \mu_i\\
  M=\sequn{\mu}}
{\squn{\ut{fsm\_inst}} \xrightarrow{} \mathcal{H}=\ttuple{M}{C}}

\step Rule \textsc{Inst} gives the interpretation of a single FSM instance as an automaton.

\newcommand{\tysequm}[3]{\langle#1_1\tm{:}#2_1,\ldots;#1_#3\tm{:}#2_#3\rangle}

\infrule[Inst]
{\senv{M}(\cat{id}) = \ttttttuple{\tysequm{i'}{\tau'}{m}}{\tysequm{o'}{\tau''}{n}}{Q}{V}{T}{\ttuple{q_0}{\vec{a_0}}}\\
  \Phi=\{i'_1 \mapsto i_1, \ldots, i'_m \mapsto i_m, o'_1 \mapsto o_1, \ldots, \o'_n \mapsto o_n \}\\
  \falln{i}{1}{m}\, \senv{I}(i_i)=\ttuple{\cat{cat}_i}{\tau_i},\ \cat{cat}_i \in
  \{\tm{input},\tm{shared}\} \wedge \tau_i=\tau'_i\\
  \falln{i}{1}{n}\, \senv{I}(o_i)=\ttuple{\cat{cat}_i}{\tau_i},\ \cat{cat}_i \in
  \{\tm{output},\tm{shared}\} \wedge \tau_i=\tau''_i\\
 \mm'=\ttttttuple{\tysequm{i}{\tau'}{m}}{\tysequm{o}{\tau''}{n}}{Q}{V}{\Phi_T(T)}{\ttuple{q_0}{\Phi_A(\vec{a_0})}}\\
 \mu=\tttuple{\mm'}{q_0}{\mathcal{I}(V)}}
{\senv{M},\senv{I} \vdash\ \tm{fsm}~\cat{id}~\sequm{i}{m}~\sequm{o}{n} \xrightarrow{} \mu}

Rule \textsc{Inst} checks the arity and the type conformance of the inputs and outputs supplied to
the instanciated model. The rule builds a \emph{substitution} $\Phi$ for binding \emph{local} input and
output names to \emph{global} ones. This substitution is applied to each transition (including the
initial one) of the resulting automaton~ using the derived functions $\Phi_T$ and $\Phi_A$ (not
detailed here).
% \begin{eqnarray*}
%  \Phi_T(\{\tau_1,\ldots,\tau_n\}) & = & \{ \Phi_\tau(\tau_1), \ldots, \Phi_\tau(\tau_n) \}\\
%  \Phi_\tau(\ttttuple{q}{\ttuple{e}{\{exp_1,\ldots,exp_m\}}}{\sequn{a}}{q'}) & = & 
% \ttttuple{q}{\ttuple{\Phi(e)}{\{\Phi_E(exp_1),\ldots,\Phi_E(exp_m)\}}}{\sequn{\Phi_A(a)}}{q'})
% \end{eqnarray*}
The $\mathcal{I}$ function builds an environment from a set of names, initializing each
  binding with the $\bot$ (``undefined'') value~:
  $$
  \mathcal{I}(\setn{x}) = \{x_1 \mapsto \bot, \ldots, x_n \mapsto \bot \}
  $$

\section{Dynamic semantic}
\label{sec:dynamic-semantics}

The dynamic semantics of a \textsc{Core Rfsm} program will be given in terms of (instantaneous)
\textbf{reactions}

\begin{eqnarray*}
  \mathcal{C} \vdash\ M,\ \env \xrightarrow[\rho]{\sigma} M',\ \env' 
\end{eqnarray*}

meaning

\begin{center}
``in the static context $\mathcal{C}$ and given a (dynamic) environment $\Gamma$, a set of
automata $M$ reacts to a \emph{stimulus} $\sigma$ leading to an updated set of automata $M'$, an
updated environment $\Gamma'$ and producing a \emph{response} $\rho$''
\end{center}

\subsection*{Definitions}
\label{sec:dynsem-defs}

\step Given an expression $\expr$ and an environment $\env$, $\eval{\env}{\expr}$ denotes the value obtained by
\textbf{evaluating} expression $\expr$ within environment $\env$. For example

$$
\eval{\{x\mapsto 1,y\mapsto 2\}}{x+y} = 3
$$

\medskip \step An \textbf{event} $e$ is either the occurrence of a \emph{pure event}
$\epsilon$ or the assignation of a \textbf{value} \emph{v} to a \textbf{name} (input, output or
local variable)~:

\begin{equation*}
  e = \begin{cases}
             \epsilon \\
             x \gets v
             \end{cases}
\end{equation*}

\medskip
\step An \textbf{event set} $E$ is a dated set of events

\begin{eqnarray*}
  E = \ttuple{t}{\setn{e}}
\end{eqnarray*}

where $t$ gives the occurrence \textbf{time} (logical instant).

\medskip
For example $E=\ttuple{10}{\{h,e \gets 0\}}$ means

\begin{center}
``At time t=10, event \texttt{h} occurs and (input) $e$ is set to 0'' .
\end{center}

\medskip
The union of \emph{event sets} is defined as
  \begin{eqnarray*}
    \ttuple{t}{e} \cup \ttuple{t'}{e'} =
    \begin{cases}
      \ttuple{t}{e \cup e'} \qquad \text{if}\ t=t' \\
      \bot\qquad \text{otherwise}
    \end{cases}
  \end{eqnarray*}

\medskip\step A \textbf{stimulus} $\sigma$ (resp. \textbf{response} $\rho$) is just an event set involving inputs
  (resp. outputs).
\todo{Input et output par rapport à quoi : automate ou contexte ?}

% \medskip
% \step A \textbf{program} $M$ is a set of automata

% \begin{equation*}
%   M = \setn{\mu}
% \end{equation*}

\subsection*{Rules}
\label{sec:dynsem-rules}

\step
Given a static description $\mathcal{H}=\ttuple{M}{C}$ of a program, the \textbf{execution} of this
program submitted to a sequence of stimuli $\vec{\sigma}=\sigma_1,\ldots,\sigma_n$ is formalized by
rule \textsc{Exec} 

% Rule \textsc{Exec} formalizes the execution of a program the reaction of a set of automata $M$
% in response to a sequence of stimuli $\sigma_1,\ldots,\sigma_n$

\infrule[Exec]
{  C \vdash\ M \xrightarrow{} M_0,\ \env_0\\
  \falln{i}{1}{n}\quad C \vdash\ M_{i-1},\ \env_{i-1} \xrightarrow[\rho_i]{\sigma_i} M_i,\ \env_i}
{\mathcal{H}=\ttuple{M}{C} \xrightarrow[\vec{\rho}=\sequn{\rho}]{\vec{\sigma}=\sequn{\sigma}} M_n,\ \env_n}

In other words, the execution of the program is described as as a sequence of \textbf{instantaneous
reactions}, which can be denoted as\footnote{Omitting context $C$, which is constant during an
execution.}~:

\begin{eqnarray*}
  M_0,\ \env_0 \xrightarrow[\rho_1]{\sigma_1} M_1,\ \env_1 \rightarrow \ldots
  \xrightarrow[\rho_n]{\sigma_n} M_n,\ \env_n
\end{eqnarray*}

\noindent where
\begin{itemize}
\item the \emph{global environment} $\Gamma$ here records the value of inputs and shared
  variables\footnote{This environment is required to handle events describing modifications of these
    values, as described below (see rule \textsc{ReactUpd}).},
\item $\rho_1, \ldots, \rho_n$ is the sequence of responses,
  \item $M_n$ and $\Gamma_n$ respectively give the final state of the automata and global
    environment.
\end{itemize}
  
\medskip \step
Rule \textsc{Init} describes how the initial set of automata $M_0$ and global environment
$\env_0$ are initialized by executing the initial transition of each automaton (producing a set of
initial responses $\rho_0$. 

\infrule[Init]
{\falln{i}{1}{n} \mu_i=\tttuple{\mm_i}{q_i}{\vars_i}\quad
  \mm_i=\ttttttuple{.}{.}{.}{.}{.}{\ttuple{.}{\vec{a_i}}}\quad
  C \vdash\ \vars_i,\ \gamma_{i-1} \xrightarrow[\ttuple{0}{\emptyset}]{\vec{a_i},0} \vars'_i,\ \gamma_i\quad
  \mu'_i=\tttuple{\mm_i}{q_i}{\vars'_i}\\
  C=\ttttttuple{.}{I_v}{.}{O_v}{.}{H_v}\quad
  \gamma_0 = \mathcal{I}(I_v \cup O_v \cup H_v)}
{C \vdash\ M=\setn{\mu} \xrightarrow{} M_0=\setn{\mu'},\ \env_0=\gamma_n}


\noindent
where the $I_v$, $O_v$ and $H_v$ sets, taken from the static context $C$, respectively give the name of
  inputs, outputs and shared variables.

\medskip
\textbf{Note}. Rule \textsc{Init} does \emph{not} produce any response
$\rho$. This is because the initial actions of an automaton cannot emit events hence can only update the
its local environment or the global one.

\medskip \step Rule \textsc{Acts} describes how a sequence of actions $\vec{a}$ (at time
$t$) updates the local and global environments, possibly emitting a set of responses\footnote{This
  set of responses is always empty when rule \textsc{Acts} is invoked in the context of \textsc{Init}.}.

\infrule[Acts]
{ \falln{i}{1}{n}\quad C \vdash\ \vars_{i-1},\ \env_{i-1} \xrightarrow[\rho_i]{a_i,\ t} \vars_i,\ \env_i\\
  \vars_0=\vars\quad \env_0=\env \quad \rho_e = \cuppn{\rho_i}}
{C \vdash\ \vars,\ \env \xrightarrow[\rho_e]{\sequn{a},\ t} \vars_n,\ \env_n}

\medskip \textbf{Note}. The definition of rule \textsc{Acts} given above enforces a \emph{sequential
  interpretation} of actions. For example
$$
\{x \mapsto 1,\ s \mapsto \bot \},\ \env \xrightarrow{\ssequ{x \gets x+1}{s \gets x},t}
\{x \mapsto 2,\ s \mapsto 2 \}, \env
$$
Rule \textsc{Acts} could easily be reformulated to describe other interpretations, such as a
\emph{synchronous} one, in which all RHS values are first evaluated
and then assigned to LHS in parallel\footnote{As happens in hardware synchronous implementations for
  example.}.
% for example~:
% $$
% \{x \mapsto 1,\ s \mapsto \bot \},\ \env \xrightarrow{\ttuple{\vec{a}}{d}}_A
% \{x \mapsto 2,\ s \mapsto 1 \}, \env
% $$


\medskip \step Rules \textsc{ActUpdL} and \textsc{ActUpdG} respectively describe the
effect of an action updating a local or global variable (shared variable or output)\footnote{The
  effect of an action emitting an event will be described by rules \textsc{ActEmitS} and \textsc{ActEmitG},
  given latter.}.

\begin{multicols}{2}
\infrule[ActUpdL]
{x \in \text{dom}(\vars) \quad v = \eval{\vars\cup\env}{\expr}}
{C \vdash\ \vars,\ \env \xrightarrow[\ttuple{t}{\emptyset}]{x\gets \expr,\ t} \vars[x\mapsto v],\ \env}

\infrule[ActUpdG]
{x \in \text{dom}(\env) \quad v = \eval{\vars\cup\env}{\expr}}
{C \vdash\ \vars,\ \env \xrightarrow[\ttuple{t}{\emptyset}]{x\gets \expr,\ t} \vars,\ \env[x\mapsto v]}
\end{multicols}

\step Rule \textsc{React} describes how a program $M$ within a global environment $\env$
(instantaneously) reacts to a stimulus (event set) $\sigma$, producing a response (event set)
$\rho$, an updated program $M'$ and an updated environment $\env'$.

\infrule[React]
{\sigma_e,\sigma_v = \Sigma(\sigma)\qquad
 C \vdash\ M,\ \env \xrightarrow{\sigma_v} M,\ \env_v\qquad
 C \vdash\ M,\ \env_v \xrightarrow[\rho_e]{\sigma_e} M',\ \env'}
{C \vdash\ M,\ \env \xrightarrow[\rho_e]{\sigma} M',\ \env'}

where the function $\Sigma$ partitions a \emph{event set} into one containing the stimuli
corresponding to \emph{pure} events ($\epsilon$) and another containing those corresponding to updates to
global inputs~:

\begin{equation*}
  \Sigma(\ttuple{t}{\setn{e}}) = \ttuple{t}{\{ e_i ~|~ e_i = \epsilon_i \}},\quad\ttuple{t}{\{
        e_i ~|~ e_i = x_i \gets v_i \}}
\end{equation*}

\step Rule \textsc{ReactUpd} describes how a program $M$ within a global environment $\env$ reacts to set
of events describing updates to global inputs. These updates are just recorded in the environment and do not produce
responses, nor trigger any reaction of the automata~:

\infrule[ReactUpd]
{}
{C \vdash\ M,\ \env \xrightarrow{\sigma_v=\ttuple{t}{\{x_1\gets v_1,\ldots,x_m\gets v_m\}}} M,\ \env[x_1\mapsto
  v_1]\ldots[x_m\mapsto v_m]}

\step
Rule \textsc{ReactEv} describes how a program reacts to a set of pure events.

\infrule[ReactEv]
{\falln{i}{1}{n} C \vdash\ \mu_{\pi(i)},\ \env_{i-1} \xrightarrow[\rho_i]{\sigma_i} \mu'_{\pi(i)},\ \env_i\qquad
 \sigma_i=\sigma_{i-1} \cup \rho_i\\
 \env_0 = \env\qquad 
 \sigma_0 = \sigma_e\qquad
 \rho_e = \cuppn{\rho_i}\qquad
 \env'=\env_n}
{C \vdash\ M=\setn{\mu}, \env \xrightarrow[\rho]{\sigma_e=\ttuple{t}{\{\epsilon_1,\ldots,\epsilon_m\}}}_e
  M'=\setn{\mu'},\ \env'}

Each automaton reacts
separately but \emph{in a specific order}. This order is derived from the dependencies between
automata. We say that an automaton $\mu'$ \emph{depends on} another automaton $\mu$ at a given
instant $t$, and note

\begin{eqnarray*}
  \mu \leq \mu'
\end{eqnarray*}

if the reaction of $\mu$ at this instant can trigger or modify the reaction of $\mu'$
at the same instant. Concretely, this happens when $\mu$ and $\mu'$ are respectively in states $q$
and $q'$ and there's (at least) one pair of transitions $(\tau,\tau')$ starting respectively from
$q$ and $q'$ so that
\begin{itemize}
\item $\tau'$ is triggered by an event emitted by $\tau$, or
\item a variable occuring in the guards associated to $\tau'$ is written by the actions associated
  to $\tau$.
\end{itemize}

The function $\pi$ used in \textsc{ReactEv} is a permutation of $\{1,\ldots,n\}$ defined so that
\begin{eqnarray*}
  \mu_{\pi(1)} \leq \mu_{\pi(2)} \leq \ldots \leq \mu_{\pi(n)}
\end{eqnarray*}

Having the automata of $M$ react in the order $\pi(1),\ldots,\pi(n)$ ensures that any event emitted or local
variable update performed by an automaton during a given reaction is effectively perceived by any
other automaton \emph{at the same reaction}, a principle called \emph{instantaneous broadcats}.

The permutation $\pi$ can easily be computed by a \emph{topological sort} of the dependency graph
derived from the conditions expressed above. In practice, this will be carried out by a static
analysis of the program. 

\step Rules \textsc{React1}, \textsc{React0} and \textsc{ReactN} describe how a single automaton reacts to a
set of pure events, updating both its internal and global states and producing another set of (pure)
events in response.

\begin{multicols}{2}
\infrule[React1]
{ \Delta_{\env \cup \vars}(\mm,q,e) = \{ \tau \}\\
  C \vdash\ \mu,\ \env \xrightarrow[\rho_e]{\tau,\ t} \mu', \env'}
{C \vdash\ \mu=\tttuple{\mm}{q}{\vars},\ \env \xrightarrow[\rho_e]{\sigma_e=\ttuple{t}{e}} \mu',\ \env'}

\infrule[React0]
{ \Delta_{\env \cup \vars}(\mm,q,e) = \emptyset}
{C \vdash\ \mu=\tttuple{\mm}{q}{\vars},\ \env \xrightarrow[\ttuple{t}{\emptyset}]{\sigma_e=\ttuple{t}{e}} \mu,\ \env}
\end{multicols}

\infrule[ReactN]
{ \Delta_{\env \cup \vars}(\mm,q,e) = \setn{\tau}\\
  \tau = \mathsf{choice}(\setn{\tau})\\
  C \vdash\ \mu,\ \env \xrightarrow[\rho_e]{\tau,\ t} \mu', \env'}
{C \vdash\ \mu=\tttuple{\mm}{q}{\vars},\ \env \xrightarrow[\rho_e]{\sigma_e=\ttuple{t}{e}} \mu',\ \env'}

Given a automaton modelised by $\mm$ and currently in state $q$, $\Delta_\env(\mm,q,e)$, where
$e=\setn{\epsilon}$, returns
the set of \emph{fireable} transitions, \emph{i.e.} all the transitions triggered by the event set
$e$ starting from $q$ and for which the all the associated boolean guards evaluate, in environment $\env$, to
\textsf{true}.

\begin{equation*}
  \Delta_\env(\mathcal{M},q,\setn{\epsilon}) = \cupp{i=1}{n}{\Delta_\env(\mathcal{M},q,\epsilon_i)}
\end{equation*}
where
\begin{equation*}
 \Delta_\env(\ttttttuple{.}{.}{.}{.}{T}{.},q,\epsilon) = \{ (q_s,c,a,q_d) \in T ~|~
  q=q_s ~\wedge~ c = \ttuple{\epsilon}{\setn{e}} ~\wedge~ \falln{i}{1}{n} \eval{\env}{e_i} = \mathsf{true} \}
\end{equation*}


Rule \textsc{React1} describes the case when the set of events triggers exactly \emph{one} transition
of the automaton. Its state and local variables are updated according to the actions listed in the
transition and the remaining actions are used to generated the set of responses.

Rule \textsc{React0} describes the case when the set of events does not trigger any transition. The
automaton and the global environment are left unchanged.

Rule \textsc{ReactN} describes the case when the set of events triggers more than one transition. This
situation corresponds to a non-deterministic behavior of the automaton. The
function \textsf{choice} is here used to choose one transition\footnote{This can done, for example,
  by adding a \emph{priority} to each transition.}. 

\medskip \step
Rule \textsc{Trans} describes the effect of performing a transition, updating the automaton local
and global states and returning a set of (pure) events as responses.

\infrule[Trans]
{\mu=\tttuple{\mm}{q}{\vars} \qquad \tau=\ttttuple{q}{c}{\vec{a}}{q'} \\
  C \vdash\ \vars,\ \env \xrightarrow[\rho_e]{\vec{a},t}_a \vars',\ \env'\\
  \mu'=\tttuple{\mm}{q'}{\vars'}}
{C \vdash\ \mu,\ \env \xrightarrow[\rho_e]{\tau,\ t} \mu',\ \env'}

\medskip\step Rules \textsc{ActEmitS} and \textsc{ActEmitG} complement the rules \textsc{ActUpdL} and \textsc{ActUpdG}
given previously by describing the effect of an action emitting a shared or output event. The
$H_e$ set, taken from context $C$, is here used to distinguish between to to. The formers can trigger the reaction of other(s)
automaton(s), the latters are just ignored here (see note below).

\begin{multicols}{2}
\infrule[ActEmitS]
{C=\ttttttuple{.}{.}{.}{.}{H_e}{.}\quad
  \epsilon \in H_e} 
{C \vdash\ \vars,\ \env \xrightarrow[\ttuple{t}{\{\epsilon\}}]{\epsilon,\ t} \vars,\ \env}

\infrule[ActEmitG]
{C=\ttttttuple{.}{.}{.}{.}{H_e}{.}\quad
 \epsilon \not\in H_e} 
{C \vdash\ \vars,\ \env \xrightarrow[\ttuple{t}{\emptyset}]{\epsilon,\ t} \vars,\ \env}
\end{multicols}

\bigskip
\textbf{Note}.
The semantics described here only defines how a program execution progresses, from the initial
program to the final program state $M$. In practice, an interpreter will also build a \emph{trace}
of such an execution, recording all significant events (stimuli, responses, state moves,
\emph{etc.}). Building such a trace is easily performed by modifying the semantic rules given above. It
has not been done here for the sake of simplicity.

%%% Local Variables:
%%% mode: latex
%%% TeX-master: "rfsm"
%%% End:
