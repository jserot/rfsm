\chapter{ Building language variants}
\label{cha:variants}

Following the approach described in~\cite{Leroy00} for example, the RFSM compiler is implemented in
a \emph{modular} way. The language is split into a \emph{host} language, describing the general
structure and behavior of FSMs (states, transitions, \ldots) and a \emph{guest}
language\footnote{``Base language'' in the terminology of \cite{Leroy00}.} describing the syntax and
semantics of \emph{expressions} used in transition guards and actions. 
Technically, this ``separation of concern'' is realized by providing the host language in the form
of a \emph{functor} taking as argument the module implementing the guest language.

This approach makes it fairly easy to produce variants of the ``standard'' RFSM language -- with
dedicated type systems and expression languages typically -- by simply defining the module defining
the guest language and applying the aforementioned functor. Actually, the ``standard'', RFSM
  language was designed using this approach, starting from a very simple ``core'' guest
  language gradually enriched with new features at the expression level\footnote{Traces of the
    incremental design process can be found in the \texttt{srs/guests/core},
    \texttt{src/guests/others/szdints} and \texttt{src/guests/others/szvars} directories for example.}. 

The directory \verb|src/guests/templ| in the distribution provides the basic structure for deploying
this approach. In practice, to implement a language variant, one has to
\begin{itemize}
\item write the implementation of the guest language in the form of a collection of modules in the
  \texttt{lib} subdirectory\footnote{These modules will be encapsulated in a single module and the
    latter will be passed to the host functor to build the target language.},
\item write the lexer and the parser for this guest language (expressions, type expressions, \ldots),
\item build the compiler by simply invoking \emph{make}  
\end{itemize}

\medskip
To illustrate this process, we describe in the sequel the implementation of a very simple language
language for which the guest language has only two types, `event` and `bool`,
and expressions are limited to boolean constants and variables\footnote{This language is essentially
  that provided in \texttt{src/guests/others/mini}.}. We focus here on the most salient features. 
The \texttt{Readme} file in the  \texttt{templ} directory describes the procedure in
details. Other examples, given in \verb|src/guests/core| and \verb|src/guests/others|\footnote{And, of
  course, in \texttt{src/guests/std}.}, can also be used as guidelines. 

\section{Implementing the \texttt{Guest} module}

The module implementing the guest language must match the following signature~:

\begin{lstlisting}[language={[Objective]Caml},frame=single,basicstyle=\small,label={lst:guest-sig}]
module type T = sig
  module Info : INFO
  module Types : TYPES
  module Syntax : SYNTAX with module Types = Types
  module Typing : TYPING with module Syntax = Syntax and module Types = Types
  module Value : VALUE with type typ = Types.typ
  module Static : STATIC with type expr = Syntax.expr and type value = Value.t
  module Eval : EVAL with module Syntax = Syntax and module Value = Value
  module Ctask: CTASK with module Syntax = Syntax
  module Systemc: SYSTEMC with module Syntax = Syntax and module Static = Static and type value = Value.t
  module Vhdl: VHDL with module Syntax = Syntax and module Static = Static and type value = Value.t
  module Error : ERROR
  module Options : OPTIONS
end
\end{lstlisting}

\noindent
where
\begin{itemize}
\item module \texttt{Info} gives the name and version of the guest language,
\item module \texttt{Syntax} describes the (abstract) syntax of the guest language,
\item modules \texttt{Types} and \texttt{Typing} respectively describe the types and the typing
  rules of the guest language,
\item module \texttt{Static} describes the static semantics of the guest language (basically, the
  interpretation of model parameters),
\item modules \texttt{Value} and \texttt{Eval} respectively describe the values and the dynamic
  semantics manipulating these values,
\item modules \texttt{CTask}, \texttt{Systemc} and \texttt{Vhdl} respectively describe the
  guest-level part of the C, SystemC and VHDL backends,
\item module \texttt{Error} describes how guest-specific errors are handled,
\item module \texttt{Options} describes guest-specific compiler options.
\end{itemize}

Listings~\ref{lst:mini-types}, \ref{lst:mini-syntax},\ref{lst:mini-value} and \ref{lst:mini-eval}
respectively show the contents of the \texttt{Types},
\texttt{Syntax}, \texttt{Value} and \texttt{Eval} modules for the \texttt{mini} language.
The definition of non essential fonctions has been omitted\footnote{See the corresponding source files in
  \texttt{src/guests/others/mini/lib} for a complete listing.}.

\begin{lstlisting}[language={[Objective]Caml},frame=single,basicstyle=\small,caption={Module
    \texttt{Guest.Types} (excerpt)},label={lst:mini-types}]
type typ =
  | TyEvent
  | TyBool
  | TyUnknown

let no_type =  TyUnknown

let is_event_type (t: typ) = match t with TyEvent -> true | _ -> false
let is_bool_type (t: typ) =  match t with TyBool -> true | _ -> false

let pp_typ ?(abbrev=false) fmt (t: typ) = ...
\end{lstlisting}

In module \texttt{Types}, the key definition is that of type \texttt{typ}, which describes the
guest-level types, \emph{i.e.} the types which can be attributed to guest-level expressions and
variables. The type \texttt{TyUnknown} is used to define the value \texttt{no_type} which is
attributed by the host language to (yet) untyped syntax elements.

\begin{lstlisting}[language={[Objective]Caml},frame=single,basicstyle=\small,caption={Module
    \texttt{Guest.Syntax} (excerpt)}, label={lst:mini-syntax}]
module Types = Types

module Location = Rfsm.Location
module Annot = Rfsm.Annot
module Ident = Rfsm.Ident

let mk ~loc x = Annot.mk ~loc ~typ:Types.no_type x

(* Type expressions *)

type type_expr = (type_expr_desc,Types.typ) Annot.t
and type_expr_desc = TeConstr of string (* name, no args here *)

let is_bool_type (te: type_expr) = ...
let is_event_type (te: type_expr) = ...

let pp_type_expr fmt (te: type_expr) =  ...

(* Expressions *)
  
type expr = (expr_desc,Types.typ) Annot.t
and expr_desc = 
  | EVar of Ident.t
  | EBool of bool

let vars_of_expr (e: expr) = ...
and pp_expr fmt (e: expr) = ...

(* LHSs *)
  
type lhs = (lhs_desc,Types.typ) Annot.t
and lhs_desc = Ident.t

let lhs_var (l: lhs) = ...
let vars_of_lhs (l: lhs) = ...
let is_simple_lhs (l: lhs) = true

let mk_simple_lhs (v: Ident.t) =
  Annot.{ desc=v; typ=Types.no_type; loc=Location.no_location }

let pp_lhs fmt l =  ...
\end{lstlisting}

The module \texttt{Syntax} uses the modules \texttt{Location}, \texttt{Annot}  and \texttt{Ident}
provided by the \texttt{Rfsm} host library. These modules provide types and functions to handle
source code locations, syntax annotations and identifiers respectively. The type
\verb|('a,Types.typ) Annot.t| is associated to syntax nodes of type
\verb|'a|. The types \verb|type_expr|, \verb|expr| and \verb|lhs| respectively describe guest-level
type expressions, expressions and left-hand-sides (LHS). LHS are used in the definition of
actions. In this language they are limited to simple identifiers (ex: \verb|x:=<expr>|) but the
guest language can use other forms (like in the RFSM ``standard'' language which support arrays and
records in LHS). 

\begin{lstlisting}[language={[Objective]Caml},frame=single,basicstyle=\small,caption={Module
    \texttt{Guest.Values} (excerpt)},label={lst:mini-value}]
type typ =
  | TyEvent
  | TyBool
  | TyUnknown

let no_type =  TyUnknown

let is_event_type t = match t with TyEvent -> true | _ -> false
let is_bool_type t =  match t with TyBool -> true | _ -> false
let mk_type_fun ty_args ty_res = ...

let pp_typ ?(abbrev=false) fmt t = ...
\end{lstlisting}

\begin{lstlisting}[language={[Objective]Caml},frame=single,basicstyle=\small,caption={Module
    \texttt{Guest.Value} (excerpt)},label={lst:mini-value}]
type t =
  | Val_bool of bool
  | Val_unknown

let default_value ty = match ty with
  | _ -> Val_unknown

exception Unsupported_vcd of t

let vcd_type (v: t) = match v with
  | Val_bool _ -> Rfsm.Vcd_types.TyBool
  | _ -> raise (Unsupported_vcd v)

let vcd_value (v: t) = match v with
  | Val_bool v -> Rfsm.Vcd_types.Val_bool v
  | _ -> raise (Unsupported_vcd v)

let pp fmt (v: t) =  ...
\end{lstlisting}

The module \verb|Guest.Value| defines the values, associated to guest-level expressions by the
dynamic semantics. The value \verb|Val_unknown| is used to represent undefined or unitialized
value. The functions \verb|vcd_type| and \verb|vcd_value| provide the interface to the VCD backend,
generating simulation traces~: they should return a VCD compatible representation  (defined in the
host library \verb|Vcd_types| module) of a value.

\begin{lstlisting}[language={[Objective]Caml},frame=single,basicstyle=\small,caption={Module
    \texttt{Guest.Eval} (excerpt)},label={lst:mini-eval}]
module Syntax = Syntax
module Value = Value
module Env = Rfsm.Env
module Annot = Rfsm.Annot

type env = Value.t Env.t

exception Illegal_expr of Syntax.expr
exception Uninitialized of Rfsm.Location.t

let mk_env () = Env.init []

let upd_env lhs v env = Env.upd lhs.Annot.desc v env

let lookup ~loc v env = 
 match Rfsm.Env.find v env with
  | Value.Val_unknown -> raise (Uninitialized loc)
  | v -> v

let eval_expr env e = match e.Annot.desc with
  | Syntax.EVar v -> lookup ~loc:e.Annot.loc v env
  | Syntax.EBool i -> Val_bool i 

let eval_bool env e = 
  match eval_expr env e with
  | Val_bool b -> b
  | _ -> raise (Illegal_expr e) 

let pp_env fmt env = ...
\end{lstlisting}

The module \verb|Guest.Eval| defines the guest-level dynamic semantics, \emph{i.e.} the definition
of the dynamic environment \texttt{env}, used to bind identifiers to values and of the 
functions \verb|eval_expr| and \verb|eval_bool| used to evaluate guest-level expressions. The
presence of a distinct, \verb|eval_bool| function is required because the boolean type and
expressions are not part of the host-level syntax. The definition of the \texttt{env} type here uses
that provided by the RFSM host library but any definition matching the corresponding signature would do.

\section{Implementing the guest language parser}

The parser for the guest language defines the concrete syntax for guest-level type expressions,
expressions and LHS. It is written in a separate \texttt{.mly} file which will be combined with the
parser for the host language\footnote{Using \texttt{menhir} facility to split parser specifications
  into multiple files.}. There are only two guest specific tokens here, denoting the boolean
constants \texttt{true} and \texttt{false}. The \texttt{open} directive in the prologue section
gives access to the abstract syntax definitions given in \texttt{Guest.Syntax} module. The function
\texttt{mk}, defined in this module, builds annotated syntax nodes, inserting the source code
location. 

\begin{lstlisting}[language={menhir},frame=single,basicstyle=\small,caption={File
    guest_parser.mly},label={lst:mini-eval}]
%token TRUE
%token FALSE

%{
open Mini.Top.Syntax
%}

%%
%public type_expr:
  | tc = LID { mk ~loc:$sloc (TeConstr tc) }

%public lhs:
  | v = LID { mk ~loc:$sloc (mk_ident v) }

%public expr:
  | v = LID { mk ~loc:$sloc (EVar (mk_ident v)) }
  | c = scalar_const { c }

%public scalar_const:
  | TRUE { mk ~loc:$sloc (EBool true) }
  | FALSE { mk ~loc:$sloc (EBool false) }

%public const:
  | c = scalar_const { c }

%public stim_const: 
  | c = scalar_const { c }

\end{lstlisting}

\section{Implementing the guest language lexer}

The \texttt{ocamllex} tool, used for defining the lexer, does not support multi-file definitions.
The lexer for the guest-specific part of the target language is therefore supplied in the form of
code fragments to be inserted in the host lexer definition file\footnote{Technically, this insertion is
  performed using the \texttt{cppo} tool.}. These fragments are listed in two separate files, which
must be supplied in the \texttt{bin} subdirectory of the guest directory~:
\begin{itemize}
\item the file \texttt{guest_kw.mll} contains the lexer definition of the guest-specific keywords,
\item the file \texttt{guest_rules.mll} contains the guest-specific rules.
\end{itemize}

In the case of the \texttt{mini} language defined here, the latter is empty and the former only
contains the two following lines.

\begin{verbatim}
"true", TRUE;
"false", FALSE;
\end{verbatim}

\section{Building the compiler}

Defining the target language implementation and building the associated compiler is then simply
obtained by the two following functor applications\footnote{For technical reasons, these two
  statements are placed in distinct files, \texttt{bin\/lang.ml} and \texttt{bin\/rfsmc.ml}.}~:

\begin{lstlisting}[language={[Objective]caml},frame=single,basicstyle=\small]
module L = Rfsm.Host.Make(Mini.Top)

module Compiler =
  Rfsm.Compiler.Make
    (L)
    (Lexer)
    (struct include Parser type program = L.Syntax.program end)
\end{lstlisting}

Invoking the compiler now boils down to executing

\begin{lstlisting}[language={[Objective]caml},frame=single,basicstyle=\small]
let _ = Printexc.print Compiler.main ()
\end{lstlisting}

%%% Local Variables:
%%% mode: latex
%%% TeX-master: "rfsm_rm"
%%% End:
