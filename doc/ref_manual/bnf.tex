\chapter{ Formal syntax of RFSM programs}
\label{cha:bnf}

This appendix gives a BNF definition of the concrete syntax RFSM programs.
As stated in the introduction, this syntax is that of the so-called \emph{standard} RFSM language.
Variant languages will essentially differ in the definition of the
$\rfsmtypedecl*{}$,
$\rfsmtypeexpr*{}$,
$\rfsmexpr*{}$,
$\rfsmconstant*{}$,
and $\rfsmconst*{}$
syntactical categories.

\medskip
The meta-syntax is conventional. Keywords are written in \textbf{boldface}.  Non-terminals are
enclosed in angle brackets ({\tt <} \ldots {\tt >}).  Vertical bars ({\tt |}) indicate
alternatives.  Constructs enclosed in non-bold brackets ({\tt [} \ldots {\tt ]}) are optional.
The notation $E^*$ (resp $E^+$) means zero (resp one) or more repetitions of $E$, separated by spaces.
The notation $E^*_x$ (resp $E^+_x$) means zero (resp one) or more repetitions of $E$, separated by
symbol $x$. Terminals \verb|lid| and \verb|uid| respectively designate identifiers
starting with a lowercase and uppercase letter. 

\begin{rfsmgrammar}
\rfsmgramfunc{\rfsmfsmmodel{}}& \rfsmgramdef & \rfsmFSM*{} \rfsmMODEL*{}
                                               \rfsmID{}
                                               \rfsmgramopt{\rfsmparams*{}}
                                               \rfsmLPAREN*{}
                                               \rfsmgramseplist{\rfsmCOMMA*{}}{\rfsmio*{}}
                                               \rfsmRPAREN*{} \\
                                               & & \rfsmLBRACE*{} \\
                                               & & \rfsmSTATES*{} \rfsmCOLON*{} 
                                               \rfsmgramseplist{\rfsmCOMMA*{}}{\rfsmID{}} 
                                               \rfsmSEMICOLON*{} \\
                                               & & \rfsmgramopt{\rfsmvars*{}} \\
                                               & & \rfsmTRANS*{} \rfsmCOLON*{}
                                               \rfsmgramseplist{\rfsmCOMMA*{}}{\rfsmtransition*{}}
                                               \rfsmSEMICOLON*{} \\
                                               & & \rfsmITRANS*{} \rfsmCOLON*{}
                                               \rfsmitransition*{}
                                               \rfsmSEMICOLON*{} \\
                                               & & \rfsmRBRACE*{}
  \\& & \\

\rfsmgramfunc{\rfsmparams{}}& \rfsmgramdef & \rfsmLT*{}
                                             \rfsmgramseplist{\rfsmCOMMA*{}}{\rfsmparam*{}}
                                             \rfsmGT*{}
  \\& & \\

\rfsmgramfunc{\rfsmparam{}}& \rfsmgramdef & \rfsmID{} \rfsmCOLON*{}
                                            \rfsmtyp*{}
  \\& & \\

\rfsmgramfunc{\rfsmio{}}& \rfsmgramdef & \rfsmIN*{} \rfsmiodesc*{}\\
  & \rfsmgrambar &\rfsmOUT*{} \rfsmiodesc*{}
  \\& & \\

\rfsmgramfunc{\rfsmiodesc{}}& \rfsmgramdef & \rfsmID{} \rfsmCOLON*{}
                                             \rfsmtyp*{}
  \\& & \\

\rfsmgramfunc{\rfsmvars{}}& \rfsmgramdef & \rfsmVARS*{} \rfsmCOLON*{}
                                           \rfsmgramseplist{\rfsmCOMMA*{}}{\rfsmvar*{}}
                                           \rfsmSEMICOLON*{}
  \\& & \\

\rfsmgramfunc{\rfsmvar{}}& \rfsmgramdef & \rfsmID{} \rfsmCOLON*{}
                                          \rfsmtyp*{}
  \\& & \\

\rfsmgramfunc{\rfsmtransition{}}& \rfsmgramdef & \rfsmID{}
                                                 \rfsmARROWSTART*{}
                                                 \rfsmcondition*{}
                                                 \rfsmgramopt{\rfsmactions*{}}
                                                 \rfsmARROWEND*{} \rfsmID{}
  \\& & \\

\rfsmgramfunc{\rfsmitransition{}}& \rfsmgramdef & \rfsmgramopt{\rfsmactions*{}}
                                                  \rfsmARROWEND*{} \rfsmID{}
  \\& & \\

\rfsmgramfunc{\rfsmcondition{}}& \rfsmgramdef & \rfsmID{}\\
  & \rfsmgrambar &\rfsmID{} \rfsmDOT*{}
                  \rfsmgramsepnelist{\rfsmDOT*{}}{\rfsmguard*{}}
  \\& & \\

\rfsmgramfunc{\rfsmguard{}}& \rfsmgramdef & \rfsmrelexpr*{}
  \\& & \\

\rfsmgramfunc{\rfsmactions{}}& \rfsmgramdef & \rfsmBAR*{}
                                              \rfsmgramsepnelist{\rfsmSEMICOLON*{}}{\rfsmaction*{}}
  \\& & \\

\rfsmgramfunc{\rfsmaction{}}& \rfsmgramdef & \rfsmID{}\\
  & \rfsmgrambar &\rfsmID{} \rfsmCOLEQ*{} \rfsmexpr*{}
  \\& & \\

\rfsmgramfunc{\rfsmglobal{}}& \rfsmgramdef & \rfsmINPUT*{} \rfsmID{}
                                             \rfsmCOLON*{} \rfsmtyp*{}
                                             \rfsmEQUAL*{} \rfsmstimuli*{}\\
  & \rfsmgrambar &\rfsmOUTPUT*{} \rfsmID{} \rfsmCOLON*{} \rfsmtyp*{}\\
  & \rfsmgrambar &\rfsmSHARED*{} \rfsmID{} \rfsmCOLON*{} \rfsmtyp*{}
  \\& & \\

\rfsmgramfunc{\rfsmstimuli{}}& \rfsmgramdef & \rfsmPERIODIC*{} \rfsmLPAREN*{}
                                              \rfsmINT*{} \rfsmCOMMA*{}
                                              \rfsmINT*{} \rfsmCOMMA*{}
                                              \rfsmINT*{} \rfsmRPAREN*{}\\
  & \rfsmgrambar &\rfsmSPORADIC*{}
                  \rfsmLPAREN*{} \rfsmgramseplist{\rfsmCOMMA*{}}{\rfsmINT*{}} \rfsmRPAREN*{}\\
  & \rfsmgrambar &\rfsmVALUECHANGES*{}
                  \rfsmLPAREN*{} \rfsmgramseplist{\rfsmCOMMA*{}}{\rfsmvaluechange*{}} \rfsmRPAREN*{}
  \\& & \\

\rfsmgramfunc{\rfsmvaluechange{}}& \rfsmgramdef & \rfsmINT*{} \rfsmCOLON*{}
                                                  \rfsmINT*{}
  \\& & \\

\rfsmgramfunc{\rfsmfsminst{}}& \rfsmgramdef & \rfsmFSM*{} \rfsmID{}
                                              \rfsmEQUAL*{} \rfsmID{}
                                              \rfsmgramopt{\rfsmLT*{}
                                              \rfsmgramsepnelist{\rfsmCOMMA*{}}{\rfsmINT*{}}
                                              \rfsmGT*{}}
                                              \rfsmLPAREN*{}
                                              \rfsmgramseplist{\rfsmCOMMA*{}}{\rfsmID{}}
                                              \rfsmRPAREN*{}
  \\& & \\

\rfsmgramfunc{\rfsmtyp{}}& \rfsmgramdef & \rfsmTYEVENT*{}\\
  & \rfsmgrambar &\rfsmTYINT*{} \rfsmgramopt{\rfsmintrange*{}}\\
  & \rfsmgrambar &\rfsmTYBOOL*{}
  \\& & \\

\rfsmgramfunc{\rfsmintrange{}}& \rfsmgramdef & \rfsmLT*{}
                                               \rfsmtypeindexexpr*{}
                                               \rfsmDOTDOT*{}
                                               \rfsmtypeindexexpr*{}
                                               \rfsmGT*{}
  \\& & \\

\rfsmgramfunc{\rfsmtypeindexexpr{}}& \rfsmgramdef & \rfsmINT*{}\\
  & \rfsmgrambar &\rfsmID{}\\
  & \rfsmgrambar &\rfsmLPAREN*{} \rfsmtypeindexexpr*{} \rfsmRPAREN*{}\\
  & \rfsmgrambar &\rfsmtypeindexexpr*{} \rfsmPLUS*{} \rfsmtypeindexexpr*{}\\
  & \rfsmgrambar &\rfsmtypeindexexpr*{} \rfsmMINUS*{} \rfsmtypeindexexpr*{}\\
  & \rfsmgrambar &\rfsmtypeindexexpr*{} \rfsmTIMES*{} \rfsmtypeindexexpr*{}\\
  & \rfsmgrambar &\rfsmtypeindexexpr*{} \rfsmDIV*{} \rfsmtypeindexexpr*{}\\
  & \rfsmgrambar &\rfsmtypeindexexpr*{} \rfsmMOD*{} \rfsmtypeindexexpr*{}
  \\& & \\

\rfsmgramfunc{\rfsmrelexpr{}}& \rfsmgramdef & \rfsmexpr*{} \rfsmEQUAL*{}
                                              \rfsmexpr*{}\\
  & \rfsmgrambar &\rfsmexpr*{} \rfsmNOTEQUAL*{} \rfsmexpr*{}\\
  & \rfsmgrambar &\rfsmexpr*{} \rfsmGT*{} \rfsmexpr*{}\\
  & \rfsmgrambar &\rfsmexpr*{} \rfsmLT*{} \rfsmexpr*{}\\
  & \rfsmgrambar &\rfsmexpr*{} \rfsmGTE*{} \rfsmexpr*{}\\
  & \rfsmgrambar &\rfsmexpr*{} \rfsmLTE*{} \rfsmexpr*{}
  \\& & \\

\rfsmgramfunc{\rfsmexpr{}}& \rfsmgramdef & \rfsmINT*{}\\
  & \rfsmgrambar &\rfsmID{}\\
  & \rfsmgrambar &\rfsmLPAREN*{} \rfsmexpr*{} \rfsmRPAREN*{}\\
  & \rfsmgrambar &\rfsmexpr*{} \rfsmPLUS*{} \rfsmexpr*{}\\
  & \rfsmgrambar &\rfsmexpr*{} \rfsmMINUS*{} \rfsmexpr*{}\\
  & \rfsmgrambar &\rfsmexpr*{} \rfsmTIMES*{} \rfsmexpr*{}\\
  & \rfsmgrambar &\rfsmexpr*{} \rfsmDIV*{} \rfsmexpr*{}\\
  & \rfsmgrambar &\rfsmexpr*{} \rfsmMOD*{} \rfsmexpr*{}
  \\& & \\

\end{rfsmgrammar}


%%% Local Variables:
%%% mode: latex
%%% TeX-master: "rfsm"
%%% End:
