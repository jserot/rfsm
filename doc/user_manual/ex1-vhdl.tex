\chapter*{Appendix C3 - Example of generated VHDL code}  
\label{cha:ex1-vhdl}

This is the code generated from program given in Listing~\ref{lst:rfsm-gensig} by the VHDL backend.

\begin{lstlisting}[language=VHDL,frame=single,numbers=none,basicstyle=\small,caption=File g.vhd]
library ieee;
use ieee.std_logic_1164.all;
use work.rfsm.all;

entity G is
  port( H: in std_logic;
        E: in std_logic;
        S: out std_logic;
        rst: in std_logic
        );
end entity;

architecture RTL of G is
  type t_state is ( E0, E1 );
  signal state: t_state;
begin
  process(rst, H)
  variable k: integer;
  begin
    if ( rst='1' ) then
      state <= E0;
      S <= '0';
    elsif rising_edge(H) then 
      case state is
      when E0 =>
        if ( E = '1' ) then
          k := 1;
          S <= '1';
          state <= E1;
        end if;
      when E1 =>
        if ( k = 3 ) then
          S <= '0';
          state <= E0;
        elsif  ( k<3 ) then
          k := k+1;
        end if;
    end case;
    end if;
  end process;
end architecture;

\end{lstlisting}

\begin{lstlisting}[language=VHDL,frame=single,numbers=none,basicstyle=\small,caption=File main_top.vhd]
library ieee;
use ieee.std_logic_1164.all;	   

entity main_top is
  port(
        H: in std_logic;
        E: in std_logic;
        S: out std_logic;
        rst: in std_logic        );
end entity;

architecture struct of main_top is

component G 
  port(
        H: in std_logic;
        E: in std_logic;
        S: out std_logic;
        rst: in std_logic
        );
end component;


begin
  G0: G port map(H,E,S,rst);
end architecture;
\end{lstlisting}

\begin{lstlisting}[language=VHDL,frame=single,numbers=none,basicstyle=\small,caption=File main_tb.vhd]
library ieee;
use ieee.std_logic_1164.all;	   

use work.rfsm.all;
entity main_tb is
end entity;

architecture struct of main_tb is

component main_top is
  port(
        H: in std_logic;
        E: in std_logic;
        S: out std_logic;
        rst: in std_logic        );
end component;

signal H: std_logic;
signal E: std_logic;
signal S: std_logic;
signal rst: std_logic;

begin

  inp_H: process
    type t_periodic is record period: time; t1: time; t2: time; end record;
    constant periodic : t_periodic := ( 10 ns, 10 ns, 80 ns );
    variable t : time := 0 ns;
    begin
      H <= '0';
      wait for periodic.t1;
      t := t + periodic.t1;
      while ( t < periodic.t2 ) loop
        H <= '1';
        wait for periodic.period/2;
        H <= '0';
        wait for periodic.period/2;
        t := t + periodic.period;
      end loop;
      wait;
  end process;
  inp_E: process
    type t_vc is record date: time; val: std_logic; end record;
    type t_vcs is array ( 0 to 2 ) of t_vc;
    constant vcs : t_vcs := ( (0 ns,'0'), (25 ns,'1'), (35 ns,'0') );
    variable i : natural := 0;
    variable t : time := 0 ns;
    begin
      for i in 0 to 2 loop
        wait for vcs(i).date-t;
        E <= vcs(i).val;
        t := vcs(i).date;
      end loop;
      wait;
  end process;
  reset: process
  begin
    rst <= '1';
    wait for 1 ns;
    rst <= '0';
    wait for 100 ns;
    wait;
  end process;

  Top: main_top port map(H,E,S,rst);

end architecture;

\end{lstlisting}

%%% Local Variables: 
%%% mode: latex
%%% TeX-master: "rfsm"
%%% End: 
