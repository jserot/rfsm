\chapter{Introduction}
\label{chap:intro}

This document is a brief user manual for the RFSM language and compiler.

\medskip
RFSM is a domain specific language aimed at describing, drawing and simulating \emph{reactive finite state
  machines}. Reactive FSMs are a FSMs for which transitions can only take place at the occurence of
events.

\medskip
RFSM has been developed mainly for pedagogical purposes, in order to initiate students to
model-based design. It is currently used in courses dedicated to embedded system design both on
software and hardware platforms (microcontrolers and FPGA resp.). But RFSM can also be used to
generate code (C, SystemC or VHDL) from high-level models to be integrated to existing applications.

More precisely, RFSM can be used to
\begin{itemize}
\item describe FSM-based models and testbenches,
\item generate graphical representations of these models (\verb|.dot| format) for visualisation,
\item simulate these models, producing \verb|.vcd| files to be displayed with waveform viewers such
  as \texttt{gtkwave},
\item generate C, SystemC and VHDL implementations (including testbenches for simulation)
\end{itemize}

\medskip
The RFSM compiler is also used internally by the \textsc{RfsmLight}
application\footnote{\url{https://github.com/jserot/rfsm-light}} which provides a GUI-based interface to a
subset of the language\footnote{Single FSM models only.} and compiler back-ends. The
\textsc{RfsmLight} application is described in a separate document.

\medskip
This document is organized as follows.
Chapter~\ref{cha:overview} is short presentation of the language and its possibilities using simple
examples. Chapter~\ref{cha:language} is a more throrough presentation describing its syntax in a more
systematic way and discussing the main issues related to its semantics.
Chapter~\ref{cha:rfsmc} describes how to use the command-line compiler. Appendices
A1, A2 and A3 give some examples of code generated by the C, SystemC and VHDL backends.

\medskip
\textbf{Note}. The language described in this document is the so-called \emph{standard} RFSM language. The
distribution\footnote{\url{https://github.com/jserot/rfsm}} actually contains several variants, all
sharing a common \emph{host} language but differing, essentially, in the so-called \emph{guest} language used to describe
the conditions and actions attached to transitions. The framework available in the distribution
can be used to build one's own variant language. This process is described in the reference manual. 

%%% Local Variables: 
%%% mode: latex
%%% TeX-master: "rfsm_um"
%%% End: 
