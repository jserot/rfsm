\chapter*{Appendix A2 - Example of generated SystemC code}  
\label{cha:ex1-systemc}

This is the code generated from program given in Listing~\ref{lst:rfsm-gensig} by the SystemC
backend.

\begin{lstlisting}[language=systemc,frame=single,numbers=none,basicstyle=\small,caption=File g.h]
#include "systemc.h"

SC_MODULE(G)
{
  // Types
  typedef enum { E0,E1 } t_state;
  // IOs
  sc_in<bool> H;
  sc_in<bool> E;
  sc_out<bool> S;
  // Local variables
  t_state state;
  int k;

  void react();

  SC_CTOR(G) {
    SC_THREAD(react);
    }
};

\end{lstlisting}

\begin{lstlisting}[language=systemc,frame=single,numbers=none,basicstyle=\small,caption=File g.cpp]
#include "g.h"
#include "rfsm.h"

void G::react()
{
  state = E0;
  S.write(false);
  while ( 1 ) {
    switch ( state ) {
    case E0:
      wait(H.posedge_event());
      if ( E.read()==true ) {
        k = 1;
        S.write(true);
        state = E1;
        }
      wait(SC_ZERO_TIME);
      break;
    case E1:
      wait(H.posedge_event());
      if ( k==3 ) {
        S.write(false);
        state = E0;
        }
      else if ( k<3 ) {
        k = k+1;
        }
      wait(SC_ZERO_TIME);
      break;
    }
  }
};
\end{lstlisting}

\begin{lstlisting}[language=systemc,frame=single,numbers=none,basicstyle=\small,caption=File inp_H.h]
#include "systemc.h"

SC_MODULE(Inp_H)
{
  // Output
  sc_out<bool> H;

  void gen();

  SC_CTOR(Inp_H) {
    SC_THREAD(gen);
    }
};
\end{lstlisting}

\begin{lstlisting}[language=systemc,frame=single,numbers=none,basicstyle=\small,caption=File inp_H.cpp]
#include "inp_H.h"
#include "rfsm.h"

typedef struct { int period; int t1; int t2; } _periodic_t;

static _periodic_t _clk = { 10, 0, 80 };

void Inp_H::gen()
{
  int _t=0;
  wait(_clk.t1, SC_NS);
  notify_ev(H,"H");
  _t = _clk.t1;
  while ( _t <= _clk.t2 ) {
    wait(_clk.period, SC_NS);
    notify_ev(H,"H");
    _t += _clk.period;
    }
};
\end{lstlisting}

\begin{lstlisting}[language=systemc,frame=single,numbers=none,basicstyle=\small,caption=File inp_E.h]
#include "systemc.h"

SC_MODULE(Inp_E)
{
  // Output
  sc_out<sc_uint<1> > E;

  void gen();

  SC_CTOR(Inp_E) {
    SC_THREAD(gen);
    }
};
\end{lstlisting}

\begin{lstlisting}[language=systemc,frame=single,numbers=none,basicstyle=\small,caption=File inp_E.cpp]
#include "inp_E.h"
#include "rfsm.h"

typedef struct { int date; int val; } _vc_t;
static _vc_t _vcs[3] = { {0,0}, {25,1}, {35,0} };

void Inp_E::gen()
{
  int _i=0, _t=0;
  while ( _i < 3 ) {
    wait(_vcs[_i].date-_t, SC_NS);
    E = _vcs[_i].val;
    _t = _vcs[_i].date;
    _i++;
    }
};
\end{lstlisting}

\begin{lstlisting}[language=systemc,frame=single,numbers=none,basicstyle=\small,caption=File main.cpp]
#include "systemc.h"
#include "rfsm.h"
#include "inp_H.h"
#include "inp_E.h"
#include "g.h"

int sc_main(int argc, char *argv[])
{
  sc_signal<bool> H;
  sc_signal<bool> E;
  sc_signal<bool> S;
  sc_trace_file *trace_file;
  trace_file = sc_create_vcd_trace_file ("main");
  sc_write_comment(trace_file, "Generated by RFSM v2.0");
  sc_trace(trace_file, H, "H");
  sc_trace(trace_file, E, "E");
  sc_trace(trace_file, S, "S");

  Inp_H Inp_H("Inp_H");
  Inp_H(H);
  Inp_E Inp_E("Inp_E");
  Inp_E(E);

  G g("g");
  g(H,E,S);

  sc_start(100, SC_NS);

  sc_close_vcd_trace_file (trace_file);

  return EXIT_SUCCESS;
}
\end{lstlisting}

%%% Local Variables: 
%%% mode: latex
%%% TeX-master: "rfsm_um"
%%% End: 
